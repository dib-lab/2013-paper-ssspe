\documentclass[11pt]{article}
\usepackage{simplemargins}
\usepackage{times}
\usepackage{graphicx}
\usepackage[small,compact]{titlesec}

\pagestyle{empty}

\title{BIGDATA: Small: DA: DCM: Low-memory Streaming Prefilters for
Biological Sequencing Data}
\author{C. Titus Brown}
\date{July 2012}

\setlength{\parindent}{0pt}
\setlength{\parskip}{0.70ex}
\setallmargins{1in}

\title{Walking the talk: adopting and adapting sustainable scientific software
  development processes in a small biology lab.}

\author{Michael R. Crusoe$^{1}$\\
C. Titus Brown$^{2,1\ast}$\\
\small \bf{1} Microbiology and Molecular Genetics,\\
\small \bf{2} Computer Science and Engineering,\\
\small Michigan State University, East Lansing, MI, USA\\
\small $\ast$ Corresponding author: ctb@msu.edu}

\begin{document}

\maketitle

\abstract{There's a funny thing about arguing that scientific
software should be better designed, better tested, and more openly
developed: one's peers may reasonably hold you to this higher
standard.  This is the story of our attempt to figure out how to
hold ourselves to these higher standards on the khmer project, 
a small, single-lab software project in biology.}

\setlength{\parindent}{0pt}
\setlength{\parindent}{0pt}
\setlength{\parskip}{0.70ex}

\section{Introduction}

In July of 2013, Michael R. Crusoe (MRC) was hired in in C. Titus
Brown's (CTB) Laboratory of Genomics, Evolution, and Development at
Michigan State University.  This brief recounts MRC's experience so
far in the lab as an engineer and budding researcher, including his
assessment of the group's main body of code; CTB's reaction and
engagement with the resulting issues; and a collaborative discussion
of how best to prioritize and pace infrastructure work, research work,
and the associated reduction (or accumulation) of technical debt.

The GED lab is in a phase of tool building, and the main body of code
is the khmer suite of nucleotide sequence analysis tools [cite], for
which MRC is now responsible.  This code is primarily aimed at the
``Big Data'' problem of efficiently preprocessing very large amounts
of data being produced by sequencing machines that have become
available in the last five years.  It is now four years old and the
basis for a published paper, a PhD thesis, an invited book chapter,
and three submitted preprints, as well as about five papers in various
stages of development.  khmer grew out of specific analysis needs, and
was developed primarily on startup funding and as part of a USDA
grant.  Its development has led to at least two additional grants,
including the NIH BIG DATA grant that now supports MRC.  Over its
lifetime khmer has had 13 different contributors, seven of which are
currently active.  khmer is largely written in C++ with a thin,
hand-coded Python wrapper; the unit and functional tests are entirely
written in Python.  (diagram@@),

\paragraph{Dramatis personae:}
CTB's background includes about 20 years of research software
development, starting from his work on the Avida project (@@wikipedia
page; 1993 Avida ALIFE paper).  In addition to research software, he
participated in a number of small open source projects, largely in
Python; from 2004-2008 his primary software development focus was on
testing, when he created several testing tools.  However, his PhD is
in developmental biology and his research focus has always been on the
intersection of domain-specific problems and effective open source
programming.

MRC's background includes computer systems engineering, free and
open-source systems, and microbiology; this combined with his prior
experience as a user of the khmer suite gave him plenty of standing to
generate opinions about how to implement CTB's goal of ``better
science through superior software.''  [cite]

\paragraph{Project history:}
The khmer project was started by CTB as part of his research when he
arrived at MSU in 2008 as an Assistant Professor.  It grew from a very
small initial core of exact substring counting functionality to
provide an implementation of a memory-efficient CountMin-Sketch data
structure (manuscript in preparation); a sequence assembly graph
implementation using Bloom filters \cite{kmer-percolation}; and an
implementation of a novel algorithm for lossy compression of data
structure \cite{diginorm}.  Its estimated user base is in the 100s,
and it is currently being extended in a number of ways (see below).
khmer started out under git version control and was developed using
``stupidity-driven development (SDD)'' a term coined by CTB in which test
development is primarily driven by the need to prevent regressions to
previously observed bugs.

Throughout the khmer project, there have been a number of conflicting
development goals: first, khmer serves as a research testbed for the
development of novel data structures and algorithms; second, khmer
also provides novel end-user functionality that enables certain types
of data analysis; and third, we are working to integrate khmer into a
larger research ecosystem of tools.  This last goal is the hardest
and requires the most collaborative outreach, because khmer
primarily provides a set of filters for preprocessing data: the input
data is generated by upstream research instruments and software
filters, and khmer's output is fed into downstream research software.
Therefore, khmer must interact correctly at both a technical and
scientific level with other software that is completely out of our
control.

Specific plans for khmer's future development include significant
improvements in its processing efficiency (@eric mcdonald posa
chapter), the implementation of novel algorithms for scaling (@cite
infinite assembly/career grant), the implementation of specific novel
algorithms for data-driven biological discovery (@cite NIH R01),
integration with existing pipelines and user interfaces (@cite USDA
grant and R01), and the extension of khmer as a general research
platform for rapid exploratory development of novel substring
approaches to biological data analysis.

In sum, we plan to optimize and scale khmer, add specific new
functionality, and integrate it with other software, all while
preventing software regression and expanding its flexibility for
as-yet unplanned future research.  These plans are simply impossible
without some sort of software development process, which (at the time
of MRC's arrival) could at best be described as ``haphazard.''

\section{Initial Evaluation and Short-term Plans}

To guide our development of a rational software development process,
we applied the Software Sustainability Institute's Criteria-based
assessment checklist [cite] to the khmer project shared the results
with the community [cite, storify the tweets?]. The summary from that
report was grim: khmer met 17 of 42 (or 40\%) of of the SSI's criteria
for Usability, and 43 of 119 (or 36\%) of the criteria for
Sustainability, for an overall fulfillment of 43 of the 119 items, or
36\%.

The low fulfillment of the SSI's criteria is slightly embarrassing for
the group for several reasons: CTB is a collaborator and co-author
with SSI; quotes from him are featured in case study presentations
[cite]; he is cited as someone who is doing things the correct way
[recomputation manifesto citation]; and he regularly criticizes the
field for poor software process.  This is essentially the point,
however: because these goals were highly valued within the lab, and
because the software filled a novel niche, CTB successfully argued for
resources to address these issues [cite R01 blog post].  Making
significant progress towards these ends isn't merely a matter of
budgets and hiring: even with a full time hybrid software
engineer/bioinformatician on the team it is not straightforward to
prioritize infrastructure work, let alone navigate between the core
research work, user support, community engagement, and collaborations.

Our guiding principle, as eloquently put by MRC's first mentor, is
``What is best for science?'' /cite{Nagy2007}. The (re)usability of
khmer suite by others and the group's own ability to continue to
extend and innovate with the codebase are two concrete areas of
concern. On the usability front the most pressing issue is the
lack of versioned releases and the fragility of the complicated build
procedure due to a mixture of C++, C, and Python.  There is an
existing body of tests but we do not know how well they cover the C++
code.  Moreover, there is no regular continuous integration (automated
extension of the existing tests) as code is committed.

We are working towards a versioned release process and we are setting
up a continuous integration system using Jenkins.  This system already
builds the codebase and runs the test; soon it will measure and report
statement coverage of both Python and C++ code.  Once we we have a
measurement of how much of the core functionality is not covered by
automated tests, we will fix those deficiencies and be able to consider
more ambitious refactoring.

\section{Medium-term Plans}

We hope to release a ``1.0'' version in April 2014, and during this period
we plan to adopt a number of new practices.

First and foremost, we will move to a ``pull request'' model for
contributions, also known as ``github flow,'' in which both small and
large changes are started on branches and then discussed, debated, and
updated in a persistent conversational form using GitHub's Pull
Request functions.  This can be used to educate contributors, require
new members of the lab to adopt good coding practice, and (perhaps
most importantly) enforce code review prior to merge.  Preliminary
experience suggest that this model is easy for long-time software
developers to adopt.

Secondly, we are working on a series of integration tests in which
khmer's functionality is tested in concert with a number of externally
developed packages.  This serves both an inward facing goal, where we
can verify that our software plays correctly with other codebases, and
an outward facing goal, where we can essentially place other codebases
on which we depend ``under test.''  While this latter goal might seem
a bit passive aggressive, in practice it is useful because so few
bioinformatics packages have any automated tests whatsoever.  

One additional challenge that we could address by instrumenting the
integration tests is to detect performance regressions in the khmer
code base.  Because khmer may run for hours, days, or weeks on very
large data sets with properties that we cannot yet model accurately,
running khmer on real data is the only way to observe the consequences
of code changes on performance.

Finally, we want to put in place a set of explicit citation requests.
We do not know how to measure and report khmer usage, although
altmetrics is beginning to provide measures of online impact.
Conservatively, paper citations seem likely to be the most practical
and realizable way of defining khmer's impact on science, and so we
will lay out distinct guidelines for appropriately citing algorithms,
software, and protocols.

\section{Persistent Challenges in Research Software Development}

In the long term, we expect to face three major challenges in continuing
to develop khmer.

First, we need to show the value of the process, so that we and others
can convince colleagues and funding agencies that software development
processes should be a first-class participant in publication and
funding considerations.  This will almost certainly have to be done
by doing a better (faster, higher quality) job of tackling core
scientific and biological questions, and will need to be measured
in publications and citations.

Second, we must continually reevaluate how much and where to refactor
and test.  In the face of changing input data (due to instrument and
experimental protocol changes), different expectations for output
(bioinformaticians invent a new format every 5 minutes on average),
competing algorithms with poor replicability, etc., we believe we
could spend 100\% of our time on quality control without developing
anything truly new.  However there is little or no reward in academia
for merely continuing to produce functioning software: software
maintenance grants are few and far between.  So our core software
maintenance efforts must instead be directed at enabling us to agilely
tackle research problems, i.e. focus on issues that we don't yet
understand or for which we don't have much intuition.  While this may
seem like a standard software engineering problem, we believe that
the nature of pure research may lead to additional challenges here.

Third, we face systemic and severe cultural challenges in terms of
recruiting software developers and researchers.  Inevitably new lab
members are undertrained in most of what we do, including testing,
version control, good computational hygiene, data science, and/or the
domain of biology.  These are a lot of subjects to train new people
in, and we have yet to establish an effective lab culture.  On the
converse side, of course, we expect lab graduates to be increasingly
employable in both academia and biotech; moreover, the lab reputation
of caring about such things has started to attract people who are
already somewhat better trained.

\section{A brief dialogue}

MRC: Before the evaluation was started, how mature did you think khmer
was as a project?

CTB: Early adolescence.  We'd demonstrated scientific value, and,
between the automated tests and the version control, I felt that we
were in the 99th percentile of bioinformatics codebases - yes, it's a
nice, low bar.  However, I knew the software was not as robust as it
needed to be to support continual incremental development and
refactoring, especially by new developers.  I knew much more work
needed to be done on the testing, in particular.

MRC: The work required to nurture a project into one that is more
sustainable is non-trivial. How do you justify those resources?

CTB: I haven't entirely figured that out.  The hope is that we can
adapt and innovate faster because we are working off of a more
stable code base.  So far, so good.  But can we scale this to a bigger
group of developers?  I don't know.

MRC: What particular challenges do you think life scientists face in
making the software artifacts of the research process reusable and
sustainable?

CTB: The two primary challenges are cultural (which was expected) and
technical (which was unexpected).  By cultural, I mean that there is
virtually no culture of computational reproducibility or software
development in biology, which makes it hard to discuss much less
justify.  Technically, the infrastructure and tools for enabling
reproducibility are still quite young and do not fit the needs of
subdomains terribly well.  For example, I thought that other fields
would have tools for provenance and workflow tracking that we could
easily adapt; this is simply not the case.  In fact, our most effective set
of tools has emerged from open source and data science work, including
the excellent Python and GitHub communities, IPython Notebook, and
cloud computing infrastructure.

MRC: If you could change one thing about the project process from the
past, what would it be?

CTB: With hindsight, I would place greater emphasis on putting some of
the ``trivial'' little scripts we use for data manipulation under test
and documentation.  I've had several experiences where bugs in those
simple scripts have caused days or weeks of stress, and the lack of
good documentation for these scripts has been by far the biggest
complaint of users.

MRC: What has been the most challenging aspect of the project, from a
process perspective?

CTB: Choosing how to allocate finite time to development, testing,
communication, fund-raising, and infrastructure -- essentially what
every startup goes through -- but in a background of blank
incomprehension or outright negativity from most of my colleagues
and administrators.

MRC: What has been the most useful change since my arrival?

CTB: I expect the adoption of ``github flow'' (@cite) to have the biggest
positive long-term impact; it really helps with visibility and discussion
of features, and it has already helped train new contributors in our
expectations.

%[et cetera]

%[Summary & conclusion]

%[funding ack for MRC

\section*{Acknowledgments}

MRC has been funded by AFRI Competitive Grant no. 2010-65205-20361
from the USDA NIFA and is now funded by the National Human Genome
Research Institute of the National Institutes of Health under Award
Number R01HG007513; both under C. Titus Brown.

\bibliographystyle{cell}
\bibliography{all}

%\bibliography{wssspe13-ged}

%"Criteria-based assessment of the khmer suite" http://goo.gl/MZGGhc 

%"Open Call for Projects" http://www.software.ac.uk/open-call

%"Institute Case Studies: From Consultancy Projects to Case Studies" http://www.software.ac.uk/attach/CaseStudies.pdf

%Wilson, G., Aruliah, D.A.,Titus Brown, C.T., Chue Hong, N. P., Davis, M., Guy, R. T., Haddock, S.H.D, Huff, K., Mitchell, I.M., Plumbley, M., Waugh, B., White, E. P. and Wilson, P. Best Practices for Scientific Computing. CoRR, arXiv:1210.0530 [cs.MS], 2012.

%"The five stars of research software" http://www.software.ac.uk/blog/2013-04-09-five-stars-research-software

%"The Recomputation Manifesto" http://www.software.ac.uk/blog/2013-07-09-recomputation-manifesto

%"Issues ged-lab/khmer" https://github.com/ged-lab/khmer/issues?direction=desc&milestone=1&page=1&sort=updated&state=open

%'"Stupidity Driven Testing" and PyCon \'07' http://ivory.idyll.org/blog/stupidity-driven-testing.html

\end{document}
